\documentclass[a4paper,14pt]{extarticle}

% Пакеты для работы с русским языком и кодировкой
\usepackage[utf8]{inputenc} % Кодировка UTF-8
\usepackage[T2A]{fontenc}
\usepackage[russian]{babel} % Русский язык
\usepackage{amsmath, amssymb} % Для математических формул
\usepackage{geometry} % Устанавливаем поля
\geometry{left=2cm,right=2cm,top=2cm,bottom=2cm}
\usepackage[hidelinks]{hyperref} % Для гиперссылок
\usepackage{graphicx} % Для вставки изображений
\usepackage{listings} % Для красивого отображения кода
\usepackage{listingsutf8} % Добавляет поддержку UTF-8 символов в окружении listings
\usepackage{xcolor} % Для дополнительной настройки цветов
\usepackage{tcolorbox}
\usepackage{varwidth}

\lstset{
	backgroundcolor=\color{black!78!white},     % Чёрный фон
	basicstyle=\ttfamily\color{white}, % Белый моноширинный шрифт
	keywordstyle=\color{cyan},          % Цвет ключевых слов (голубой)
	commentstyle=\color{gray},         % Цвет комментариев (серый)
	stringstyle=\color{green!90!black},% Цвет строк (зелёный)
	showstringspaces=false,            % Не показывать пробелы внутри строк
	numbers=left,                      % Номера строк слева
	numberstyle=\color{black},         % Стиль номеров строк (чёрный)
	stepnumber=1,                      % Нумерация через одну строку
	numbersep=5pt,                     % Расстояние между номерами строк и кодом
	frame=single,                      % Рамка вокруг кода
	rulecolor=\color{white},           % Цвет рамки
	breaklines=true,                   % Перенос длинных строк
	breakatwhitespace=true,            % Переносить строки только на пробелах
	captionpos=b,                      % Заголовок рамки снизу
	escapeinside={\%*}{*)},            % Использование LaTeX внутри кода
	extendedchars=true,                % Поддержка расширенных символов
	inputencoding=utf8                 % Кодировка входного текста
}

% Определение CSS
\lstdefinelanguage{CSS}{
	morekeywords={
		@import, @media, @keyframes, @font-face, @page, 
		border, margin, padding, display, position, color, background-color, width, height, 
		left, right, top, bottom, font-size, font-family, font-weight, text-align, text-decoration,
		transform, rotate, translateX, opacity
	},
	sensitive=true, % Учитывать регистр символов
	morecomment=[l][\color{gray}]{/*}, % Комментарии в CSS
	morestring=[b][\color{orange}]" % Строки в CSS
}

% Настройка tcolorbox для API
\tcbuselibrary{listingsutf8} % Поддержка кода внутри tcolorbox
\tcbset{colframe=gray!80, colback=gray!10, coltitle=black, fonttitle=\bfseries}

% Стиль для GET-запросов
\newtcolorbox[auto counter, number within=section]{getapi}[2][]{
	colframe=blue!75,
	colback=blue!5,
	colbacktitle=blue!15,
	fonttitle=\bfseries,
	title=GET: #2,
	listing only,
	listing options={basicstyle=\ttfamily\footnotesize,breaklines=true},
	#1
}

% Определяем стиль для POST-запросов
\newtcolorbox[auto counter, number within=section]{postapi}[2][]{
	title=POST: #2,
	colframe=green!75,
	colback=green!5,
	colbacktitle=green!15,
	fonttitle=\bfseries,
	listing only,
	listing options={basicstyle=\ttfamily\footnotesize,breaklines=true},
	#1
}

% Определяем стиль для DELETE-запросов
\newtcolorbox[auto counter, number within=section]{deleteapi}[2][]{
	title=DELETE: #2,
	colframe=red!75,
	colback=red!5,
	colbacktitle=red!15,
	fonttitle=\bfseries,
	listing only,
	listing options={basicstyle=\ttfamily\footnotesize,breaklines=true},
	#1
}

% Пример для PUT-запросов
\newtcolorbox[auto counter, number within=section]{putapi}[2][]{
	title=PUT: #2,
	colframe=orange!75,
	colback=orange!5,
	colbacktitle=orange!15,
	fonttitle=\bfseries,
	listing only,
	listing options={basicstyle=\ttfamily\footnotesize,breaklines=true},
	#1
}

\begin{document}
	\begin{titlepage}
		\centering
		
		\includegraphics{FEFU-logo}\\
		МИНИСТЕРСТВО НАУКИ И ВЫСШЕГО ОБРАЗОВАНИЯ\\РОССИЙСКОЙ ФЕДЕРАЦИИ\\
		Федеральное государственное автономное образовательное учреждение высшего образования\\
		\textbf{«Дальневосточный федеральный университет»}\\
		\textbf{(ДВФУ)}
		\vspace*{0.3cm}
		\rule{\linewidth}{0.7mm}
		\vspace*{0.3cm}
		
		\textbf{ИНСТИТУТ МАТЕМАТИКИ И КОМПЬЮТЕРНЫХ ТЕХНОЛОГИЙ}\\
		\vspace*{0.3cm}
		\textbf{Кафедра математического и компьютерного моделирования}\\
		\vspace*{0.9cm}
		
		Кулахсзян Сергей Грайрович, Талмазан Анастасия Александровна\\
		\textbf{ПРОЕКТ ПО WEB-ПРОГРАММИРОВАНИЮ}\\
		\vspace*{0.3cm}
		\textbf{КУРСОВОЙ ПРОЕКТ}\\
		\vspace*{0.3cm}
		
		Направление подготовки 02.03.01сцт Сквозные цифровые технологии,\\ бакалавриатская программа «Математика и компьютерные науки»\\ Очной (заочной) формы обучения\\
		\vspace{1cm}
		
		\raggedleft
		\textbf{Студенты группы Б9123-02.03.01сцт}\\
		\rule{4.5cm}{0.3mm} С. Г. Кулахсзян\\
		\rule{4.55cm}{0.3mm} А. А. Талмазан\\
		\vspace{1cm}
		
		\textbf{Руководитель проекта}\\
		\rule{4.5cm}{0.3mm} И. С. Дроздова\\
		
		\vfill
		\centering
		Владивосток\\
		2025
	\end{titlepage}
	
	% Оглавление
	\tableofcontents
	\newpage
	
	% Введение
	\section{Введение}
	
	Данный проект представляет собой веб-приложение "Магазин видеоигр", разработанное с использованием серверного фреймворка Django на языке программирования Python и клиентской библиотеки React на языке программирования JavaScript.
	
	Основная цель проекта — создание удобного онлайн-магазина, где пользователи могут просматривать, выбирать и приобретать видеоигры. 
	
	{Основные задачи проекта:
	\begin{itemize}
		\item Реализовать серверную часть на Django для обработки запросов, работы с базой данных и предоставления API;
		\item Разработать клиентскую часть на React для создания удобного пользовательского интерфейса;
		\item Настроить взаимодействие фронтенда и бэкенда через REST API;
		\item Реализовать авторизацию, аутентификацию пользователя, а также возможность выхода из аккаунта и его удаление;
		\item Получить обширную информацию о видеоиграх, странах и основных валютах, соответсвующие стране, и курсе валют для формирования базы данных и дальнейшего применения этой информации;
		\item Добавить платёжную систему для совершения сделки купли-продажи поользователем продукта;
		\item Добавить возможность создания заявлений пользователем;
	\end{itemize}
	
	\newpage
	
	\section{Frontend}
	\subsection{Математическое приложение}
	\subsubsection{Конвертирование валюты}
	
	В базе данных цены на игры хранятся в центах, это позволяет избегать хранения информации в дробных числах, из-за чего расчёты будут более точными.
	
	У каждого пользователя цены на игры будут отображаться в той валюте, которая будет основной в стране, которую он выбрал при регистрации. Для корректного отображения необходимо конвертировать центы в корректную валюту. Это делается по формуле:\\
	
	\[
	S_{\text{currency}} = S_{\text{USD}} \times K_{\text{currency}},
	\]
	\[
	S_{\text{USD}} = \frac{S_{\text{cents}}}{100},
	\]
	
	где $S_{\text{cents}}$ -- значение цены в центах, $S_{\text{USD}}$ -- значение в долларах, $K_{\text{currency}}$ -- соотношение курса избранной валюты к доллару, $S_{\text{currency}}$ -- значение цены в избранной валюте.\\
	
	\subsubsection{Расчёт возраста}
	
	У всех игр есть возрастной рейтинг, благодаря которому определяется минимальный возраст, допустимый для приобретения продукта, согласно законодательству. Для того, чтобы знать при каких случах стоит предотвращать доступ к пользователя к определённому товару, необходимо рассчитать возраст пользователя.
	
	Возраст рассчитывается в полных годах и зависит от текущей даты. Для этого используется следующая формула:
	
	\[
	A = Y_{\text{today}} - Y_{\text{birth}},
	\]
	
	где $Y_{\text{today}}$ -- текущий год, $Y_{\text{birth}}$ -- год рождения пользователя, $A$ -- возраст пользователя в полных годах.
	
	Однако необходимо учитывать, что если день и месяц текущей даты ещё не достигли дня и месяца рождения пользователя в текущем году, возраст должен быть уменьшен на единицу. Условие корректировки записывается следующим образом:
	
	\[
	A = A - 1, \quad \text{если } (M_{\text{today}} < M_{\text{birth}}) \, \text{или} \, (M_{\text{today}} = M_{\text{birth}} \, \text{и} \, D_{\text{today}} < D_{\text{birth}}),
	\]
	
	где $M_{\text{today}}$ -- текущий месяц, $M_{\text{birth}}$ -- месяц рождения пользователя, $D_{\text{today}}$ -- текущее число, $D_{\text{birth}}$ -- число рождения пользователя.\\
	
	\subsubsection{Анимация загрузки}
	
	Часто на сайте изображения долго грузятся, дабы не было никакого пустого пространиства на сайте, было принято решение добавить анимацию загрузки, которая представляет из себя движение 8 кругов по окружности друг за другом.
	
	Создан компонент, представляющий собой анимацию, в которой 8 кругов вращаются вокруг центра по окружности, начиная с разных позиций и плавно появляется. Рассмотрим работу этого компонента более детально.
	
	Для создания 8 элементов (кругов) используется следующий цикл:
	
	\[
	Array.from({\ length: 8 }).map((\_, \text{index}) \Rightarrow \ldots)
	\]
	
	Этот массив из 8 элементов проходит цикл, и для каждого индекса генерируется элемент типа \texttt{div} с уникальными стилями. Основная цель заключается в том, чтобы отобразить 8 кругов, которые будут расположены равномерно по окружности.
	
	Каждому кругу присваивается угловая трансформация, которая зависит от его индекса в массиве. Стиль для каждого круга выглядит так:
	
	\[
	transform: rotate\left(\frac{360}{8} \times \text{index}\right)deg, \text{translateX}(150\%), \text{rotate}\left(-\frac{360}{8} \times \text{index}\right)deg
	\]
	
	Где:
	\begin{itemize}
		\item $rotate\left(\frac{360}{8} \times \text{index}\right)$ — этот угол позволяет расположить каждый круг на равном расстоянии друг от друга по окружности (360 градусов делятся на 8 частей).
		\item $\text{translateX}(150\%)$ — этот параметр сдвигает круг от его исходной позиции на 150\% от его радиуса, таким образом круги будут располагаться по внешней окружности.
		\item $rotate\left(-\frac{360}{8} \times \text{index}\right)$ — поворачивает каждый круг на противоположный угол, чтобы сохранить правильную ориентацию элементов.
	\end{itemize}
	
	Таким образом, каждый круг будет располагаться на окружности, и их движение будет синхронизировано.
	
	Для анимации вращения каждого круга используется следующий CSS-код:	

	\begin{lstlisting}[language=CSS]
@keyframes orbit {
    0% {
        transform: rotate(0deg) translateX(150%) rotate(0deg);
    }
    100% {
    transform: rotate(360deg) translateX(150%) rotate(-360deg);
    }
}
	\end{lstlisting}
	
	В начале анимации (\texttt{0\%}) каждый круг находится в своей начальной позиции. В конце анимации (\texttt{100\%}) круг проходит полный оборот (360 градусов) вокруг центра и возвращается на своё начальное место.
	
	Для плавного появления каждого круга используется анимация изменения прозрачности:
	
	\begin{lstlisting}[language=CSS]
@keyframes fade-in {
    0% {
        opacity: 0;
    }
    100% {
        opacity: 1;
    }
}
	\end{lstlisting}
	
	Каждому кругу присваивается задержка анимации, которая зависит от его индекса:
	
	\begin{lstlisting}[language=CSS]
animation: orbit 1.5s linear infinite, fade-in 0.19s {index} * 0.19s forwards;
    \end{lstlisting}
	
	Анимация вращения (\texttt{orbit}) длится 1.5 секунды и выполняется бесконечно. Анимация появления (\texttt{fade-in}) имеет задержку, которая зависит от индекса элемента, создавая эффект последовательного появления элементов.
	
	\subsection{Хранение пользовательских данных}
	Полсле регистрации или авторизации для корректной работы сервиса необходимо в каком-то виде хранить информацию о пользователе. Для этого я применяю токены.
	
	\subsubsection{Определение}
	JSON Web Token (JWT) представляет собой компактный и безопасный способ передачи данных между сторонами в виде JSON-объекта. Он широко используется для аутентификации и авторизации в веб-приложениях.
	
	\subsubsection{Структура JWT}
	JWT состоит из трёх частей, разделённых точками:
	\begin{itemize}
		\item \textbf{Header (заголовок)}
		\item \textbf{Payload (полезная нагрузка)}
		\item \textbf{Signature (подпись)}
	\end{itemize}
	Пример структуры JWT:\\
	\texttt{eyJhbGciOiJIUzI1NiIsInR5cCI6IkpXVCJ9.eyJ1c2VySWQiOiIxMjM0NTY3OD\\
	kwIiwibmFtZSI6IkpvaG4gRG9lIiwicm9sZXMiOlsiYWRtaW4iLCJ1c2VyIl19.S\\
	flKxwRJSMeKKF2QT4fwpMeJf36POk6yJV\_adQssw5c}\\
	Каждая часть кодируется в формате Base64 URL:
	\begin{itemize}
		\item \textbf{Header}: содержит информацию о типе токена (\texttt{JWT}) и алгоритме подписи (например, \texttt{HS256}).
		\item \textbf{Payload}: включает данные о пользователе и дополнительные мета-данные (например, роли, ID пользователя, срок действия токена).
		\item \textbf{Signature}: создаётся с использованием алгоритма подписи и секретного ключа для проверки подлинности токена.
	\end{itemize}
	\newpage

	\section{Backend}
	\subsection{Структура базы данных}
	\includegraphics[width=\textwidth]{database-schema}
	
	\subsection{Подключенные инструменты}
	\subsubsection{Simple JWT}
	Для аутентификации пользователей нак Django мне необходим инструмент для работы с токенами. Одним из популярных инструментов для этого является пакет django-rest-framework-simple-jwt, который реализует аутентификацию с использованием JSON Web Token (JWT). Simple JWT предоставляет эндпоинты для получения и обновления токенов.
	
	В данном проекте Simple JWT имеет следующие настройки:
	
	\begin{lstlisting}[language=Python]
SIMPLE_JWT = {
    'ACCESS_TOKEN_LIFETIME': timedelta(minutes=5),
    'REFRESH_TOKEN_LIFETIME': timedelta(days=50),
    'ROTATE_REFRESH_TOKENS': True,
    'BLACKLIST_AFTER_ROTATION': True,
    'UPDATE_LAST_LOGIN': False,
    
    'ALGORITHM': 'HS256',
    'VERIFYING_KEY': None,
    'AUDIENCE': None,
    'ISSUER': None,
    'JWK_URL': None,
    'LEEWAY': 0,
    
    'AUTH_HEADER_TYPES': ('Bearer',),
    'AUTH_HEADER_NAME': 'HTTP_AUTHORIZATION',
    'USER_ID_FIELD': 'id',
    'USER_ID_CLAIM': 'user_id',
    'USER_AUTHENTICATION_RULE': 'rest_framework_simplejwt.authentication.default_
user_authentication_rule',
    
    'AUTH_TOKEN_CLASSES': ('rest_framework_simplejwt.tokens.AccessToken',),
    'TOKEN_TYPE_CLAIM': 'token_type',
    'TOKEN_USER_CLASS': 'rest_framework_simplejwt.models.TokenUser',
    'JTI_CLAIM': 'jti',
    
    'SLIDING_TOKEN_REFRESH_EXP_CLAIM': 'refresh_exp',
    'SLIDING_TOKEN_LIFETIME': timedelta(minutes=5),
    'SLIDING_TOKEN_REFRESH_LIFETIME': timedelta(days=1),
}
	\end{lstlisting}
	
	\begin{itemize}
		\item \textbf{Жизненный цикл токенов:}
		\begin{itemize}
			\item \textbf{ACCESS\_TOKEN\_LIFETIME} -- время жизни access-токена 5 минут;
			\item \textbf{REFRESH\_TOKEN\_LIFETIME} -- время жизни refresh-токена 50 дней;
			\item \textbf{ROTATE\_REFRESH\_TOKENS} -- при обновлении access-токена создаётся новый refresh-токен;
			\item \textbf{BLACKLIST\_AFTER\_ROTATION} -- старые refresh-токены попадают в чёрный список после обновления, это сделано для большей безопасности;
			\item \textbf{SLIDING\_TOKEN\_LIFETIME} -- время жизни sliding-токена 5 минут;
			\item \textbf{SLIDING\_TOKEN\_REFRESH\_LIFETIME} -- время жизни sliding refresh-токена 1 день.
		\end{itemize}
		
		\item \textbf{Криптографические параметры:}
		\begin{itemize}
			\item \textbf{ALGORITHM} -- алгоритм шифрования токена (HMAC с SHA-256). HMAC с SHA-256 -- это алгоритм для создания криптографических подписей. Он берёт данные токена, добавляет к нему секретный ключ, применяет хэш-функцию SHA-256, в итоге на выходе получается HMAC (Hash-based Message Authentication Code) — цифровая подпись, защищающая данные от подделки;
			\item \textbf{VERIFYING\_KEY} -- ключ для верификации токена;
			\item \textbf{JWK\_URL} -- URL для загрузки JSON Web Key.
		\end{itemize}
		
		\item \textbf{Аутентификация и авторизация:}
		\begin{itemize}
			\item \textbf{AUTH\_HEADER\_TYPES} -- тип заголовка авторизации;
			\item \textbf{AUTH\_HEADER\_NAME} -- имя заголовка с токеном;
			\item \textbf{USER\_ID\_FIELD} -- поле, идентифицирующее пользователя;
			\item \textbf{USER\_ID\_CLAIM} -- claim (ключ в payload токена), в котором передаётся user\_id;
			\item \textbf{USER\_AUTHENTICATION\_RULE} -- правило аутентификации пользователей;
			\item \textbf{TOKEN\_TYPE\_CLAIM} -- claim, указывающий тип токена;
			\item \textbf{TOKEN\_USER\_CLASS} -- класс пользователя, связанного с токеном;
			\item \textbf{JTI\_CLAIM} -- уникальный идентификатор токена (используется для blacklist).
		\end{itemize}
		
		\item \textbf{Дополнительные параметры:}
		\begin{itemize}
			\item \textbf{UPDATE\_LAST\_LOGIN} -- не обновлять поле last\_login при использовании токена;
			\item \textbf{LEEWAY} -- дополнительное время для валидации токена (учитывает возможную разницу времени между серверами).
		\end{itemize}
	\end{itemize}
	
	\subsubsection{Simple Mail Transfer Protocol}
	SMTP (Simple Mail Transfer Protocol) — это протокол для отправки электронной почты. Данный протокол применяется в проекте для отправки электронных писем пользователям о коде подтвеждения или для почтовой рассылки.
	
	В качестве почтового хоста в проекте применяется smtp.gmail.com, также для данного проекта была создана специальная электронная почта gamestore.django.app@gmail.com, от которого совершаются в отправки писем.
	
	\subsubsection{Celery}
	Celery — это асинхронный менеджер задач для Python. Он позволяет выполнять фоновые задачи (например, обработку данных, отправку email) без блокировки основного приложения. Для работы с Django применяется пакет django-celery-beat. Он создаёт новые таблицы в базе данных для создания с разными временными периодами выполнения.
	
	Celery необходим брокер для хранении информации о задаче, в проекте в роли брокера используется Reddis. Для работы celery необходимо запустить 2 модуля: beat и worker. Celery beat просматривает информацию в специальной таблице базы данных для поиска задач, которые необходимо выполнить, после нахождения задачи, он передаёт в Reddis эту задачу. Celery worker в свою очередь выполняет все задачи, которые были переданы в Reddis.
	
	Всего в проекте 2 задачи. Первая -- send\_subscription\_emails. Эта задача имеет периодичность выполнения раз в день. Она находит всех пользователей, которые имеют статус "подписан", после чего отправляет всем найденным пользователям почтовую россылку. Вторая -- delete\_old\_users. Данная задача каждый день ищет пользователей со статусом "удалённый" и проверяет дату приобретения этого статуса, пользователь получил статус "удалённый" более 30 дней назад, то информация о пользователе стирается из базы данных.
	
	\subsubsection{Stripe API}
	Stripe API -- это мощный API для работы с платежами, подписками и финансами в веб-приложениях. В проекте он применяется, как со стороны фронтэнда, так и бекэнда. 
	
	В клиентской части создаются специальные компоненты для ввода данных о банковской карте пользователя. После эти данные специальным образом конвертируются для безопасности сохранности данных.
	
	Таким образом, в серверную часть уже передаются данные не самой карты, а некоторая информация, которую Stripe расшифровывает для последующей работы с ним. После совершается транзакция, а приобретённый продукт передаётся пользователю.
	
	\subsection{Роуты для обращения к API}
	Серверная часть сайта разделён на 6 приложений, каждая из которых выполняет свою задачу: currency, feedback, games, library, payments, users.
	
	\subsubsection{currency}
	\begin{getapi}{/currency/currency\_rates/}
		\textbf{Описание:} Возвращает все объекты модели CurrencyRate.\medskip
		
		\textbf{Тело ответа:}
		\begin{lstlisting}[language=Python]
[
    {
        "currency_code": "USD",
        "rate": "1.000000",
        "updated_at": "2024-12-02T12:55:54.013000Z"
    },
    ...
]
		\end{lstlisting}
	\end{getapi}
	
	\begin{getapi}{/currency/currency\_rates/get/\{currency\_code\}/}
		\textbf{Описание:} Возвращает определённый объект модели CurrencyRate в соответсвии с currency\_code.\medskip
		
		\textbf{Тело ответа:}
		\begin{lstlisting}[language=Python]
{
	"currency_code": "USD",
	"rate": "1.000000",
	"updated_at": "2024-12-02T12:55:54.013000Z"
}
		\end{lstlisting}
	\end{getapi}
	
	\begin{getapi}{/currency/country/}
		\textbf{Описание:} Возвращает все объекты модели Country.\medskip
		
		\textbf{Тело ответа:}
		\begin{lstlisting}[language=Python]
[
    {
	    "numeric_code": 520,
	    "name_ru": "Haypy",
	    "currency_symbol": "$",
	    "currency": "AUD"
    },
    ...
]
		\end{lstlisting}
	\end{getapi}
	
	\begin{getapi}{/currency/country/get/\{numeric\_code\}/}
		\textbf{Описание:} Возвращает определённый объект модели Country в соответсвии с numeric\_code.\medskip
		
		\textbf{Тело ответа:}
		\begin{lstlisting}[language=Python]
{
	"numeric_code": 520,
	"name_ru": "Haypy",
	"currency_symbol": "$",
	"currency": "AUD"
}
		\end{lstlisting}
	\end{getapi}

	\subsubsection{feedback}	
	\begin{getapi}{/feedback/feedback/}
		\textbf{Описание:} Возвращает все объекты модели Feedback, доступно только администраторам.\medskip
		
		\textbf{Тело ответа:}
		\begin{lstlisting}[language=Python]
[
    {
        "id": 29,
        "theme": "Some theme",
        "text": "Some text",
        "file": "http://127.0.0.1:8000/static/files/file.
file",
        "status": "Some status",
        "created_at": "2025-01-26T10:52:30.333019Z",
        "updated_at": "2025-01-26T16:04:07.857400Z",
        "user": 1,
        "comment": "Some comment"
    },
    ...
]
		\end{lstlisting}
	\end{getapi}
	
	\begin{getapi}{/feedback/feedback/get/\{user\_id\}/}
		\textbf{Описание:} Возвращает все объекты модели Feedback отправленные определённым пользователем в соответсвии с указанным ID пользователя user\_id.\medskip
		
		\textbf{Тело ответа:}
		\begin{lstlisting}[language=Python]
[
    {
        "id": 29,
        "theme": "Some theme",
        "text": "Some text",
        "file": "http://127.0.0.1:8000/static/files/file.
file",
        "status": "Some status",
        "created_at": "2025-01-26T10:52:30.333019Z",
        "updated_at": "2025-01-26T16:04:07.857400Z",
        "user": 1,
        "comment": "Some comment"
    },
    ...
]
		\end{lstlisting}
	\end{getapi}
	
	\begin{getapi}{/feedback/feedback/get\_by\_id/\{feedback\_id\}/}
		\textbf{Описание:} Возвращает определённый объект модели Feedback в соответсвии с feedback\_id.\medskip
		
		\textbf{Тело ответа:}
		\begin{lstlisting}[language=Python]
{
    "id": 29,
    "theme": "Some theme",
    "text": "Some text",
    "file": "http://127.0.0.1:8000/static/files/file.file",
    "status": "Some status",
    "created_at": "2025-01-26T10:52:30.333019Z",
    "updated_at": "2025-01-26T16:04:07.857400Z",
    "user": 1,
    "comment": "Some comment"
}
		\end{lstlisting}
	\end{getapi}
	
	\begin{postapi}{/feedback/create/\{user\_id\}/}
		\textbf{Описание:} Создаёт новый объект модели Feedback от пользователя в соответсвии с user\_id.\medskip
		
		\textbf{Тело запроса:}
		\begin{lstlisting}[language=Python]
{
    "theme": "Some theme",
    "text": "Some text",
    "file": "file.file"
}
		\end{lstlisting}
		
		\textbf{Тело ответа в случае успеха:}
		\begin{lstlisting}[language=Python]
{
    "message": "Feedback created"
}
		\end{lstlisting}
		
		\textbf{Тело ответа в случае провала:}
		\begin{lstlisting}[language=Python]
{
    "details": "Some error"
}
		\end{lstlisting}
	\end{postapi}
	
	\begin{putapi}{/feedback/feedback/update/\{feedback\_id\}/}
		\textbf{Описание:} Обновляет данные объекта модели Feedback в соответствии с feedback\_id.\medskip
		
		\textbf{Тело запроса:}
		\begin{lstlisting}[language=Python]
{
    "theme": "Some theme",
    "text": "Some text",
    "file": "file.file"
}
		\end{lstlisting}
		
		\textbf{Тело ответа в случае успеха:}
		\begin{lstlisting}[language=Python]
{
    "message": "Feedback updated"
}
		\end{lstlisting}
		
		\textbf{Тело ответа в случае провала:}
		\begin{lstlisting}[language=Python]
{
    "details": "Some error"
}
		\end{lstlisting}
	\end{putapi}
	
	\begin{deleteapi}{/feedback/feedback/delete/\{feedback\_id\}/}
		\textbf{Описание:} Удаляет данные объекта модели Feedback в соответствии с feedback\_id.\medskip
		
		\textbf{Тело ответа в случае успеха:}
		\begin{lstlisting}[language=Python]
{
    "message": "Feedback deleted"
}
		\end{lstlisting}
		
		\textbf{Тело ответа в случае провала из-за статуса, не равному "Отправлено":}
		\begin{lstlisting}[language=Python]
{
    "message": "Feedback can't be deleted"
}
		\end{lstlisting}
		
		\textbf{Тело ответа в случае других причин провала:}
		\begin{lstlisting}[language=Python]
{
    "details": "Some error"
}
		\end{lstlisting}
	\end{deleteapi}
	
	\subsubsection{games}
	\begin{getapi}{/games/platform/}
		\textbf{Описание:} Возвращает все объекты модели Platform.\medskip
		
		\textbf{Тело ответа:}
		\begin{lstlisting}[language=Python]
[
    {
        "id": 4,
        "name": "Windows",
        "icon": "http://127.0.0.1:8000/static/icons/windo
ws-icon.png"
    },
    ...
]
		\end{lstlisting}
	\end{getapi}
	
	\begin{getapi}{/games/platform/get/\{platform\_id\}/}
		\textbf{Описание:} Возвращает определённый объект модели Platform в соответствии с platform\_id.\medskip
		
		\textbf{Тело ответа:}
		\begin{lstlisting}[language=Python]
{
    "id": 4,
    "name": "Windows",
    "icon": "http://127.0.0.1:8000/static/icons/windows-i
con.png"
}
		\end{lstlisting}
	\end{getapi}
	
	\begin{getapi}{/games/esrb\_rating/}
		\textbf{Описание:} Возвращает все объекты модели ESRBRating.\medskip
		
		\textbf{Тело ответа:}
		\begin{lstlisting}[language=Python]
[
    {
        "id": 1,
        "name_en": "Everyone 10 and older",
        "name_ru": "C 10 let",
        "image": "http://127.0.0.1:8000/static/icons/ESRB-
Everyone-10.png",
        "minimum_age": 10
    },
    ...
]
		\end{lstlisting}
	\end{getapi}
	
	\begin{getapi}{/games/esrb\_rating/get/\{esrb\_rating\_id\}/}
		\textbf{Описание:} Возвращает определённый объект модели ESRBRating в соответствии с esrb\_rating\_id.\medskip
		
		\textbf{Тело ответа:}
		\begin{lstlisting}[language=Python]
{
    "id": 1,
    "name_en": "Everyone 10 and older",
    "name_ru": "C 10 let",
    "image": "http://127.0.0.1:8000/static/icons/ESRB-Every
one-10.png",
    "minimum_age": 10
}
		\end{lstlisting}
	\end{getapi}
	
	\begin{getapi}{/games/genre/}
		\textbf{Описание:} Возвращает все объекты модели Genre.\medskip
		
		\textbf{Тело ответа:}
		\begin{lstlisting}[language=Python]
[
    {
        "id": 1,
        "name": "Racing"
    },
    ...
]
		\end{lstlisting}
	\end{getapi}
	
	\begin{getapi}{/games/genre/get/\{genre\_id\}/}
		\textbf{Описание:} Возвращает определённый объект модели Genre в соответствии с genre\_id.\medskip
		
		\textbf{Тело ответа:}
		\begin{lstlisting}[language=Python]
{
    "id": 1,
    "name": "Racing"
}
		\end{lstlisting}
	\end{getapi}
	
	\begin{getapi}{/games/tag/}
		\textbf{Описание:} Возвращает все объекты модели Tag.\medskip
		
		\textbf{Тело ответа:}
		\begin{lstlisting}[language=Python]
[
    {
        "id": 1,
        "name": "Survival"
    },
    ...
]
		\end{lstlisting}
	\end{getapi}
	
	\begin{getapi}{/games/tag/get/\{tag\_id\}/}
		\textbf{Описание:} Возвращает определённый объект модели Tag в соответствии с tag\_id.\medskip
		
		\textbf{Тело ответа:}
		\begin{lstlisting}[language=Python]
{
    "id": 1,
    "name": "Survival"
}
		\end{lstlisting}
	\end{getapi}
	
	\begin{getapi}{/games/developer/}
		\textbf{Описание:} Возвращает все объекты модели Developer.\medskip
		
		\textbf{Тело ответа:}
		\begin{lstlisting}[language=Python]
[
    {
        "id": 10,
        "name": "Rockstar Games"
    },
    ...
]
		\end{lstlisting}
	\end{getapi}
	
	\begin{getapi}{/games/developer/get/\{developer\_id\}/}
		\textbf{Описание:} Возвращает определённый объект модели Developer в соответствии с developer\_id.\medskip
		
		\textbf{Тело ответа:}
		\begin{lstlisting}[language=Python]
{
    "id": 10,
    "name": "Rockstar Games"
}
		\end{lstlisting}
	\end{getapi}
	
	\begin{getapi}{/games/screenshot/}
		\textbf{Описание:} Возвращает все объекты модели Screenshot.\medskip
		
		\textbf{Тело ответа:}
		\begin{lstlisting}[language=Python]
[
    {
        "id": 1827221,
        "image": "https://media.rawg.io/media/screenshots/
a7c/a7c43871a54bed6573a6a429451564ef.jpg"
    },
    ...
]
		\end{lstlisting}
	\end{getapi}
	
	\begin{getapi}{/games/screenshot/get/\{screenshot\_id\}/}
		\textbf{Описание:} Возвращает определённый объект модели Screenshot в соответствии с screenshot\_id.\medskip
		
		\textbf{Тело ответа:}
		\begin{lstlisting}[language=Python]
{
    "id": 1827221,
    "image": "https://media.rawg.io/media/screenshots/a7c/
a7c43871a54bed6573a6a429451564ef.jpg"
}
		\end{lstlisting}
	\end{getapi}
	
	\begin{getapi}{\texttt{/games/game/?page=\{page\_number\}}}
		\textbf{Описание:} Возвращает все объекты модели Game с пагинацией на странице page\_number, на каждой странице 39 объектов.\medskip
		
		\textbf{Тело ответа:}
		\begin{lstlisting}[language=Python]
{
    "total_count": 9999,
    "total_pages": 257,
    "page": 2,
    "next": "http://127.0.0.1:8000/games/game/?page=3",
    "previous": "http://127.0.0.1:8000/games/game/",
    "results": [
    {
    	"id": 19710,
    	"name": "Half-Life 2: Episode One",
    	"background_image": "https://media.rawg.io/media/games/7a2/7a
2500ee8b2c0e1ff268bb4479463dea.jpg",
    	"description": "<p>Description...</p>",
    	"esrb_rating": 6,
    	"release_date": "2006-06-01",
    	"price_in_cents": 0,
    	"platforms": [4, 5, 6],
    	"genres": [2, 4],
    	"tags": [8, 13, 30, 31, 32, 42, 62, 63, 110, 111, 118, 119, 172, 193, 232, 319, 11669, 40833, 40834, 40839, 40845, 40847, 40849],
    	"screenshots": [185831, 185832, 185833, 185834, 185835],
    	"developers": [1612, 23342]
    },
    ...
}
		\end{lstlisting}
	\end{getapi}
	
	\begin{getapi}{\texttt{/games/games/get/\{user\_id\}/?page=\{page\_number\}}}
		\textbf{Описание:} Возвращает определённый объект модели Game, которые отсутствуют в библиотеке у пользователя с ID user\_id, с пагинацией на странице page\_number, на каждой странице 39 объектов.\medskip
		
		\textbf{Необязательный параметры:}
		\begin{itemize}
			\item \texttt{[\&platforms=\{platform\_id1\}[\&platforms=\{platform\_id2\}[...]]]} -- параметры для фильтрации игр по платформам, возвращает только те игры, где в качесте платформы указан хотя бы один из перечисленных \texttt{platform\_id1[, platform\_id2[, ...]]};
			\item \texttt{[\&genres=\{genre\_id1\}[\&genres=\{genre\_id2\}[...]]]} -- параметры для фильтрации игр по жанрам, возвращает только те игры, где в качесте жанра указан хотя бы один из перечисленных \texttt{genre\_id1[, genre\_id2[, ...]]};
			\item \texttt{[\&tags=\{tag\_id1\}[\&tags=\{tag\_id2\}[...]]]} -- параметры для фильтрации игр по тэгам, возвращает только те игры, где в качесте тэга указан хотя бы один из перечисленных \texttt{tag\_id1[, tag\_id2[, ...]]};
			\item \texttt{[\&developers=\{developer\_id1\}[\&developers=\{developer\_id2\} [...]]]} -- параметры для фильтрации игр по разработчикам, возвращает только те игры, где в качесте разработчиков указан хотя бы один из перечисленных \texttt{developer\_id1[, developer\_id2[, ...]]};
			\item \texttt{\&name=\{game\_title\}} -- параметр для фильтрации игр по названию, возвращает только те игры, где название включает в себя game\_title.
		\end{itemize}
		
		\textbf{Тело ответа:}
		\begin{lstlisting}[language=Python]
{
    "total_count": 9999,
    "total_pages": 257,
    "page": 2,
    "next": "http://127.0.0.1:8000/games/game/?page=3",
    "previous": "http://127.0.0.1:8000/games/game/",
    "results": [
    {...
		\end{lstlisting}
	\end{getapi}
	
	\begin{getapi}{\texttt{/games/games/get/\{user\_id\}/?page=\{page\_number\}}}
		\textbf{Продолжение...}
		\begin{lstlisting}[language=Python]
...
    {
        "id": 19710,
        "name": "Half-Life 2: Episode One",
        "background_image": "https://media.rawg.io/media/games/7a2/7a
2500ee8b2c0e1ff268bb4479463dea.jpg",
        "description": "<p>Description...</p>",
        "esrb_rating": 6,
        "release_date": "2006-06-01",
        "price_in_cents": 0,
        "platforms": [4, 5, 6],
        "genres": [2, 4],
        "tags": [8, 13, 30, 31, 32, 42, 62, 63, 110, 111, 118, 119, 172, 193, 232, 319, 11669, 40833, 40834, 40839, 40845, 40847, 40849],
        "screenshots": [185831, 185832, 185833, 185834, 185835],
        "developers": [1612, 23342]
    },
    ...
}
		\end{lstlisting}
	\end{getapi}
	
	\begin{getapi}{/games/game/get/\{game\_id\}}
		\textbf{Описание:} Возвращает определённый объект модели Game в соответствии с game\_id.\medskip
		
		\textbf{Тело ответа:}
		\begin{lstlisting}[language=Python]
{
    "id": 19710,
    "name": "Half-Life 2: Episode One",
    "background_image": "https://media.rawg.io/media/games/7a2/7a2500
ee8b2c0e1ff268bb4479463dea.jpg",
    "description": "<p>Description...</p>",
    "esrb_rating": 6,
    "release_date": "2006-06-01",
    "price_in_cents": 0,
    "platforms": [4, 5, 6],
    "genres": [2, 4],
    "tags": [8, 13, 30, 31, 32, 42, 62, 63, 110, 111, 118, 119, 172, 193, 232, 319, 11669, 40833, 40834, 40839, 40845, 40847, 40849],
    "screenshots": [185831, 185832, 185833, 185834, 185835],
    "developers": [1612, 23342]
}
		\end{lstlisting}
	\end{getapi}
	
	\begin{getapi}{/games/random\_games/\{user\_id\}}
		\textbf{Описание:} Возвращает 6 случайных объекта модели Game, которые отстутствуют в библиотеке пользователся с ID user\_id.\medskip
		
		\textbf{Тело ответа:}
		\begin{lstlisting}[language=Python]
[
    {
        "id": 19710,
        "name": "Half-Life 2: Episode One",
        "background_image": "https://media.rawg.io/media/games/7a2/7a
2500ee8b2c0e1ff268bb4479463dea.jpg",
        "description": "<p>Description...</p>",
        "esrb_rating": 6,
        "release_date": "2006-06-01",
        "price_in_cents": 0,
        "platforms": [4, 5, 6],
        "genres": [2, 4],
        "tags": [8, 13, 30, 31, 32, 42, 62, 63, 110, 111, 118, 119, 172, 193, 232, 319, 11669, 40833, 40834, 40839, 40845, 40847, 40849],
        "screenshots": [185831, 185832, 185833, 185834, 185835],
        "developers": [1612, 23342]
    },
    ...
]
		\end{lstlisting}
	\end{getapi}
	
	\begin{getapi}{/games/check\_game\_in\_library/\{user\_id\}/\{game\_id\}}
		\textbf{Описание:} Возвращает true или false в зависимости от того, есть ли у пользователя с ID user\_id в библиотеке игра с ID game\_id.\medskip
		
		\textbf{Тело ответа в случае, если есть:}
		\begin{lstlisting}[language=Python]
{
    "in_library": true
}
		\end{lstlisting}
		
		\textbf{Тело ответа в случае, если нет:}
		\begin{lstlisting}[language=Python]
{
    "in_library": false
}
		\end{lstlisting}
	\end{getapi}
	
	\begin{getapi}{/games/requirement/}
		\textbf{Описание:} Возвращает все объекты модели Requirement.\medskip
		
		\textbf{Тело ответа:}
		\begin{lstlisting}[language=Python]
[
    {
        "id": 1,
        "minimum": "Minimum:OS: Windows 10 64 Bit, Windows 8.1 64 Bit, Windows 8 64...",
        "recommended": "Recommended:OS: Windows 10 64 Bit, Windows 8.1 64 Bit, Windows...",
        "platform": 4,
        "game": 3498
    },
    ...
]
		\end{lstlisting}
	\end{getapi}
	
	\begin{getapi}{/games/requirement/get/\{game\_id\}/}
		\textbf{Описание:} Возвращает все объекты модели Requirement игры с ID game\_id.\medskip
		
		\textbf{Тело ответа:}
		\begin{lstlisting}[language=Python]
[
    {
        "id": 1,
        "minimum": "Minimum:OS: Windows 10 64 Bit, Windows 8.1 64 Bit, Windows 8 64...",
        "recommended": "Recommended:OS: Windows 10 64 Bit, Windows 8.1 64 Bit, Windows...",
        "platform": 4,
        "game": 3498
    },
    ...
]
		\end{lstlisting}
	\end{getapi}
	
	\subsubsection{library}
	\begin{getapi}{/library/library/}
		\textbf{Описание:} Возвращает все объекты модели Library.\medskip
		
		\textbf{Тело ответа:}
		\begin{lstlisting}[language=Python]
[
    {
        "user": 1,
        "game": 50677
    },
    ...
]
		\end{lstlisting}
	\end{getapi}
	
	\begin{getapi}{/library/library/get/\{user\_id\}/}
		\textbf{Описание:} Возвращает все объекты модели Library пользователя с ID user\_id.\medskip
		
		\textbf{Тело ответа:}
		\begin{lstlisting}[language=Python]
[
    {
        "user": 1,
        "game": 50677
    },
    ...
]
		\end{lstlisting}
	\end{getapi}
	
	\begin{postapi}{/library/library/add/\{user\_id\}/\{game\_id\}/}
		\textbf{Описание:} Создаёт новый объект модели Library с игрой ID game\_id для пользователя с ID user\_id.\medskip
		
		\textbf{Тело ответа в случае успеха:}
		\begin{lstlisting}[language=Python]
{
    "message": "Game added to library"
}
\end{lstlisting}
		
		\textbf{Тело ответа в случае провала из-за наличия игры у пользователя:}
		\begin{lstlisting}[language=Python]
{
    "error": "This game is already in the library"
}
		\end{lstlisting}
	\end{postapi}
	
	\subsubsection{payments}
	\begin{postapi}{/payments/save\_stripe\_info/}
		\textbf{Оприсание:} Совершает запрос в приложение stripe для совершения транзакции.\medskip
		
		\textbf{Тело запроса:}
		\begin{lstlisting}[language=Python]
{
    "email": "some_email@mail.com",
    "payment_method_id": 999,
    "price": 999
}
		\end{lstlisting}
		
		\textbf{Тело ответа в случае, если пользователь с указанной электронной почтной впервые совершает транзакцию:}
		\begin{lstlisting}[language=Python]
{
    "message": "Success", 
    "data": {
        "customer_id": 999, 
        "extra_msg": ""
    }
}
		\end{lstlisting}
		
		\textbf{Тело ответа в случае, если пользователь с указанной электронной почтной уже совершал транзакцию:}
		\begin{lstlisting}[language=Python]
{
    "message": "Success", 
    "data": {
        "customer_id": 999, 
        "extra_msg": "Customer already existed."
    }
}
		\end{lstlisting}
	\end{postapi}
	
	\subsubsection{users}
	\begin{postapi}{/users/token/}
		\textbf{Описание:} Возвращает JSON Web Token с информацией о зарегистрированном пользователе.\medskip
		
		\textbf{Тело запроса:}
		\begin{lstlisting}[language=Python]
{
    "email": "examole@mail.com"
    "password": "example_password"
}
		\end{lstlisting}
		
		\textbf{Тело ответа в случае успеха:}
		\begin{lstlisting}[language=Python]
{
    "refresh": "eyJhbGciOiJIUzI1NiIsInR5cCI6IkpXVCJ9...",
    "access": "eyJhbGciOiJIUzI1NiIsInR5cCI6IkpXVCJ9..."
}
		\end{lstlisting}
		
		\textbf{Тело ответа в слуачае провала из-за неправильной почты или неправильного пароля:}
		\begin{lstlisting}[language=Python]
{
    "detail": "No active account found with the given credentials"
}
		\end{lstlisting}
	\end{postapi}
	
	\begin{postapi}{/users/token/refresh/}
		\textbf{Описание:} Обновляет JSON Web Token пользователя, это необходимо из-за ограниченного срока действия токена.\medskip
		
		\textbf{Тело запроса:}
		\begin{lstlisting}[language=Python]
{
    "refresh": "eyJhbGciOiJIUzI1NiIsInR5cCI6IkpXVCJ9..."
}
		\end{lstlisting}
		
		\textbf{Тело ответа в случае успеха:}
		\begin{lstlisting}[language=Python]
{
    "refresh": "eyJhbGciOiJIUzI1NiIsInR5cCI6IkpXVCJ9...",
    "access": "eyJhbGciOiJIUzI1NiIsInR5cCI6IkpXVCJ9..."
}
		\end{lstlisting}
		
		\textbf{Тело ответа в случае провала из-за некорректного refresh токена:}
		\begin{lstlisting}[language=Python]
{
    "detail": "Token is invalid or expired",
    "code": "token_not_valid"
}
		\end{lstlisting}
		
		\textbf{Тело ответа в случае провала из-за заблокированного refresh токена:}
		\begin{lstlisting}[language=Python]
{
    "detail": "Token is blacklisted",
    "code": "token_not_valid"
}
		\end{lstlisting}
	\end{postapi}
	
	\begin{postapi}{/users/register/}
		\textbf{Описание:} Создаёт новые объекты User и соответствующий Profile.\medskip
		
		\textbf{Тело запроса:}
		\begin{lstlisting}[language=Python]
{
    "username": "user",
    "email": "exaple@mail.com",
    "password": "example_password",
    "password2": "example_password",
    "first_name": "Name",
    "last_name": "Surname",
    "birthdate": 30.12.2000,
    "country": 643
}
		\end{lstlisting}
		
		\textbf{Тело ответа в случае успеха:}
		\begin{lstlisting}[language=Python]
{
    "username": "user",
    "email": "exaple@mail.com"
}
		\end{lstlisting}
		
		\textbf{Тело ответа в случае провала из-за занятых электронной почты и/или имени пользователя:}
		\begin{lstlisting}[language=Python]
{
    "username": [
        "user with this username already exists."
    ],
    "email": [
        "user with this email already exists."
    ]
}
		\end{lstlisting}
		
		\textbf{Тело ответа в случае провала из-за пустого значения одного из полей:}
		\begin{lstlisting}[language=Python]
{
    "username": [
        "This field may not be blank."
    ],
    ...
}
		\end{lstlisting}
	\end{postapi}
	
	\begin{postapi}{/users/check\_password/\{user\_id\}/}
		\textbf{Описание:} Проверяет правильность пароля, соответсвующий пользователю с ID user\_id.\medskip
		
		\textbf{Тело запроса:}
		\begin{lstlisting}[language=Python]
{
    "password": "example_password"
}
		\end{lstlisting}
		
		\textbf{Тело ответа в случае успеха:}
		\begin{lstlisting}[language=Python]
{
    "details": "Password is valid"
}
		\end{lstlisting}
		
		\textbf{Тело ответа в случае провала из-за невалидного пароля:}
		\begin{lstlisting}[language=Python]
{
    "details": "Password is invalid"
}
		\end{lstlisting}
		
		\textbf{Тело ответа в случае провала из-за отсутсвия пользователя с ID user\_id:}
		\begin{lstlisting}[language=Python]
{
    "details": "User not found"
}
		\end{lstlisting}
	\end{postapi}
	
	\begin{postapi}{/users/check\_email/}
		\textbf{Описание:} Проверяет, что переданная электронная почта не занята другим пользователем.\medskip
		
		\textbf{Тело запроса:}
		\begin{lstlisting}[language=Python]
{
    "email": "example@mail.com"
}
		\end{lstlisting}
		
		\textbf{Тело ответа в случае успеха:}
		\begin{lstlisting}[language=Python]
{
    "details": "Email does not exist"
}
		\end{lstlisting}
		
		\textbf{Тело ответа в случае провала из-за существования пользователя с данной электронной почтой:}
		\begin{lstlisting}[language=Python]
{
    "details": "Email already exists"
}
		\end{lstlisting}
		
		\textbf{Тело ответа в случае провала из-за существования пользователя с данной электронной почтой, имеющий статус "удалённый":}
		\begin{lstlisting}[language=Python]
{
    "details": "User is deleted"
}
		\end{lstlisting}
	\end{postapi}
	
	\begin{postapi}{/users/check\_username/}
		\textbf{Описание:} Проверяет, что переданное имя пользователя не занята другим пользователем.\medskip
		
		\textbf{Тело запроса:}
		\begin{lstlisting}[language=Python]
{
    "username": "username"
}
		\end{lstlisting}
		
		\textbf{Тело ответа в случае успеха:}
		\begin{lstlisting}[language=Python]
{
    "details": "Username does not exist"
}
		\end{lstlisting}
		
		\textbf{Тело ответа в случае провала из-за существования пользователя с данным именем:}
		\begin{lstlisting}[language=Python]
{
    "details": "Username already exists"
}
		\end{lstlisting}
		
		\textbf{Тело ответа в случае провала из-за существования пользователя с данным именем, имеющий статус "удалённый":}
		\begin{lstlisting}[language=Python]
{
    "details": "User is deleted"
}
		\end{lstlisting}
	\end{postapi}
	
	\begin{putapi}{/users/update/\{user\_id\}/}
		\textbf{Описание:} Обновляет данные пользователя, кроме электронной почты и пароля.\medskip
		
		\textbf{Тело запроса:}
		\begin{lstlisting}[language=Python]
{
    "username": "username",
    "password": "password",
    "first_name": "Name",
    "last_name": "Surname",
    "birthdate": 30.12.2000,
    "country": 643,
    "image": "file.file"
}
		\end{lstlisting}
		
		\textbf{Тело ответа в случае успеха:}
		\begin{lstlisting}[language=Python]
{
    "details": "Data updated"
}
		\end{lstlisting}
		
		\textbf{Тело ответа в случае провала из-за неверного пароля:}
		\begin{lstlisting}[language=Python]
{
    "details": "Invalid password"
}
		\end{lstlisting}
		
		\textbf{Тело ответа в случае провала из-за существования иного пользователя с указанным именем:}
		\begin{lstlisting}[language=Python]
{
    "details": "User with this username already exists"
}
		\end{lstlisting}
	\end{putapi}
	
	\begin{putapi}{/users/subscribe/\{user\_id\}/}
		\textbf{Описание:} Изменяет статус подписки пользователя с ID user\_id на противоположный.\medskip
		
		\textbf{Тело ответа в случае подписки:}
		\begin{lstlisting}[language=Python]
{
    "details": "User subscribed true"
}
		\end{lstlisting}
		
		\textbf{Тело ответа в случае отписки:}
		\begin{lstlisting}[language=Python]
{
    "details": "User subscribed false"
}
		\end{lstlisting}
	\end{putapi}
	
	\begin{deleteapi}{/users/delete/\{user\_id\}/}
		\textbf{Описание:} Делает статус пользователя с ID user\_id удалённым.\medskip
		
		\textbf{Тело ответа в случае успеха:}
		\begin{lstlisting}[language=Python]
{
    "detail": "User deleted"
}
		\end{lstlisting}
		
		\textbf{Тело ответа в случае провала из-за отсутсвия пользователя с указанным user\_id:}
		\begin{lstlisting}[language=Python]
{
    "detail": "User not found."
}
		\end{lstlisting}
		
		\textbf{Тело ответа в случае провала из-за запроса от иного пользователя:}
		\begin{lstlisting}[language=Python]
{
    "detail": "You can't delete stranger account."
}
		\end{lstlisting}
	\end{deleteapi}
	
	\begin{putapi}{/users/recover/}
		\textbf{Описание:} Убирает статус "удалённый" у пользователя.\medskip
		
		\textbf{Тело запроса:}
		\begin{lstlisting}[language=Python]
{
    "recover": "email or username"
}
		\end{lstlisting}
		
		\textbf{Тело ответа в случае успеха:}
		\begin{lstlisting}[language=Python]
{
    "User recovered"
}
		\end{lstlisting}
		
		\textbf{Тело ответа в случае провала из-за отсутствия пользователя с таким данным:}
		\begin{lstlisting}[language=Python]
{
    "details": "User not found"
}
		\end{lstlisting}
		
		\textbf{Тело ответа в случае провала из-за иних причин:}
		\begin{lstlisting}[language=Python]
{
    "details": "error: Some error"
}
		\end{lstlisting}
	\end{putapi}
	
	\begin{postapi}{/users/create\_confirmation\_code/}
		\textbf{Описание:} Создаёт или обновляет объект модели ConfirmationCode, соответствующий указанной электронной почте, и отправляет эллектронное письмо на эту почту с кодом подтверждения.\medskip
		
		\textbf{Тело запроса:}
		\begin{lstlisting}[language=Python]
{
    "email": "example@mail.com"
}
		\end{lstlisting}
		
		\textbf{Тело ответа в случае успеха:}
		\begin{lstlisting}[language=Python]
{
    "details": "Reset password code sent"
}
		\end{lstlisting}
		
		\textbf{Тело ответа в случае провала:}
		\begin{lstlisting}[language=Python]
{
    "details": "error: Some error"
}
		\end{lstlisting}
	\end{postapi}
	
	\begin{postapi}{/users/check\_confirmation\_code/}
		\textbf{Описание:} Проверяет код подтверждения, соответствующий электронной почте.\medskip
		
		\textbf{Тело запроса:}
		\begin{lstlisting}[language=Python]
{
    "email": "example@mail.com",
    "code": "1234"
}
		\end{lstlisting}
		
		\textbf{Тело ответа в случае успеха:}
		\begin{lstlisting}[language=Python]
{
    "details": "Confirmation code is valid"
}
		\end{lstlisting}
		
		\textbf{Тело ответа в случае провала из-за невалидного кода:}
		\begin{lstlisting}[language=Python]
{
    "details": "Confirmation code is invalid"
}
		\end{lstlisting}
		
		\textbf{Тело ответа в случае провала из-за отсутствия кода подтверждения для указанной электронной почты:}
		\begin{lstlisting}[language=Python]
{
    "details": "Confirmation code not found"
}
		\end{lstlisting}
		
		\textbf{Тело ответа в случае провала из-за иних причин:}
		\begin{lstlisting}[language=Python]
{
    "details": "error: Some error"
}
		\end{lstlisting}
	\end{postapi}
	
	\begin{putapi}{/users/reset\_password/}
		\textbf{Описание:} Обновляет пароль пользователя.\medskip
		
		\textbf{Тело запроса:}
		\begin{lstlisting}[language=Python]
{
    "email": "exaple@mail.com",
    "password": "new_password"
}
		\end{lstlisting}
		
		\textbf{Тело ответа в случае усаеха:}
		\begin{lstlisting}[language=Python]
{
    "details": "Password changed"
}
		\end{lstlisting}
		
		\textbf{Тело ответа в случае провала:}
		\begin{lstlisting}[language=Python]
{
    "details": "error: Some error"
}
		\end{lstlisting}
	\end{putapi}
	
	\begin{putapi}{/users/change\_email/\{user\_id\}/}
		\textbf{Описание:} Обновляет электронную почту пользователя в соответсвии с user\_id.\medskip
		
		\textbf{Тело запроса:}
		\begin{lstlisting}[language=Python]
{
    "email": "example@mail.com"
}
		\end{lstlisting}
		
		\textbf{Тело ответа в случае успеха:}
		\begin{lstlisting}[language=Python]
{
    "details": "Email changed"
}
		\end{lstlisting}
		
		\textbf{тело ответа в случае провала:}
		\begin{lstlisting}[language=Python]
{
    "details": f"error: Some error"
}
		\end{lstlisting}
	\end{putapi}
\end{document}
